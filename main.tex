\documentclass{article}
\usepackage{graphicx} % Required for inserting images
\usepackage{hyperref}

\title{Taller de Git}
\author{Benjamin Morales y Gabriel Urbina }
\date{May 2025}

\begin{}

\maketitle

\section{¿Quien creo este codigo?}
Este codigo fue elaborado por la profesora \textbf{Valeria Herskovic}, durante una clase de \textit{Introducción a la programación}, el dia 07 de mayo, con la finalidad de aclarar ciertas caracteristicas of the listas en Python.
\section{¿Donde lo obtuvimos?}
El codigo fue extraido de un ppt que se encontraba en el canvas de la clase anteriormente mencionada. 
\section{¿Para que se utiliza?}
El codigo esta basado en el algoritmo de la criba de eratostenes, el cual tiene como funcion encontrar todos los numeros primos entre el 2 y un numero n. Lo que hace el codigo es que se le entrega un numero n, el cual correspondera el valor hasta el cual deseamos encontrar los numeros primos, luego generara una lista llamada criba que contenga todos los valores desde el 2 hasta n (incluyendo a estos) y luego procedera a entrar en un loop en el que encontrara el numero menor de la criba (el cual es siempre el 2 en primer lugar), sacara a este numero de la criba, lo agregara a una segunda lista que correspondera a los numeros primos y luego eliminara de la criba todos los multiplos de este numero y asi hasta que la criba este finalmente vacia. Finalmente una vez acabado el loop el codigo imprimira la lista con todos los numeros encontrados en la criba.
\end{document}
